\chapter*{\textbf{LISTA DE ABREVIATURAS E SIGLAS}}
\begin{acronym}[PRODIST] %% Depois de rodar a primeira vez, digite entre os [ ] a sigla mais longa para justificar a lista de acrônimos à esquerda
   \itemsep0em
    \acro{2D}{Duas Dimensões}
    \acro{ANEEL}{Agência Nacional de Energia Elétrica}
    \acro{B}{Susceptância}
    \acro{CCEE}{Câmara de Comercialização de Energia Elétrica}
    \acro{CA}{Corrente Alternada}
    \acro{CC}{Corrente Contínua}
    \acro{EPE}{Empresa de Pesquisa Energética}
    \acro{EBP}{Equações de Balanço de Potência}
    \acro{FACTS}{Sistema de Transmissão CA Flexível}
    \acro{FP}{Fluxo de Potência}
    \acro{FPC}{Fluxo de Potência Continuado}
    \acro{FPO}{Fluxo de Potência Ótimo}
    \acro{FPMO}{Fluxo de Potência com Multiplicador Ótimo}
    \acro{FPPOL}{Fluxo de Potência em Coordenadas Polares}
    \acro{FPRET}{Fluxo de Potência em Coordenadas Retangulares}
    \acro{FPORS}{Fluxo de Potência Ótimo com Restrições de Segurança}
    \acro{FOB}{Função Objetivo}
    \acro{GS}{Gauss-Seidel}
    \acro{G}{Condutância}
    \acro{IEEE}{Institute of Electrical and Electronics Engineers}
    \acro{N}{Número de Barras do Sistema}
    \acro{NPQ}{Número de Barras PQ}
    \acro{NPV}{Número de Barras PV}
    \acro{NR}{Newton-Raphson}
    \acro{ONS}{Operador Nacional do Sistema}
    \acro{PMC}{Ponto de Máximo Carregamento}
    \acro{PQ}{Barra de Carga}
    \acro{PRODIST}{Procedimentos de Distribuição de Energia Elétrica}
    \acro{PV}{Barra de Geração}
    \acro{R}{Resistência}
    \acro{SEP}{Sistema Elétrico de Potência}
    \acrodefplural{SEP}{Sistemas Elétricos de Potência}
    \acro{SIN}{Sistema Interligado Nacional}
    \acro{p.u.}{Por Unidade}
    \acro{UFJF}{Universidade Federal de Juiz de Fora}
    \acro{X}{Reatância indutiva}
\end{acronym}