\begin{resumo}

Este trabalho apresenta uma análise sobre uma modificação do Fluxo de Potência em regime permanente utilizando o método de Newton-Raphson, o Fluxo de Potência com Multiplicador Ótimo, através da comparação com a formulação em coordenadas polares e retangulares. 

A pesquisa envolve a demonstração teórica dos métodos, bem como a implementação e testes em sistemas conhecidos na literatura: o IEEE 14 Barras, de transmissão, e o IEEE 33 Barras, de distribuição. A comparação da robustez de cada código é feita em três diferentes aspectos: tempo computacional, máximo carregamento e análise dos fractais de Newton-Raphson para cada sistema, visando compreender as vantagens e desvantagens da metodologia proposta.

O estudo indicou que o Fluxo de Potência com Multiplicador Ótimo, por mais que tenha exigido um maior tempo computacional e uma diferença irrisória na capacidade de convergir casos muito carregados, esse foi capaz de melhorar a convergência para uma maior faixa de palpites iniciais para as variáveis do problema em questão.


\textbf{Palavras-chave}: Fluxo de potência; Fractais; Newton-Raphson; Otimização; Sistemas de potência.
%%finalizadas por ponto e inicializadas por letra maiuscula.
\end{resumo}
