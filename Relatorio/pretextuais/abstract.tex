\begin{otherlanguage*}{english}
\begin{resumo}[\textbf{ABSTRACT}]
This work presents an analysis of a modification of the Newton-Raphson method for Power Flow computation, namely the Optimal Power Flow (OPF), through comparison with formulations in polar and rectangular coordinates.

The research includes theoretical demonstrations of the methods as well as their implementation and testing on well-known systems in the literature: the IEEE 14-Bus transmission system and the IEEE 33-Bus distribution system. The robustness of each algorithm is compared in three different aspects: computational time, maximum loading, and analysis of Newton-Raphson fractals for each system, aiming to understand the advantages and disadvantages of the proposed methodology.

The study indicated that the Power Flow with Optimal Multiplier, despite requiring more computational time and showing a negligible difference in the ability to converge heavily loaded cases, was able to improve convergence for a wider range of initial guesses for the problem's variables.

\textbf{Keywords}: Fractals; Newton-Raphson; Optimization; Power flow; Power Systems. 
\end{resumo}
\end{otherlanguage*}